\section{Critical Review}
Drawbacks of the Model:
\begin{enumerate}
\item Predictions do not cover full range of data, as evinced by deviations in the QQ plot
\item The independence assumption of the \texttt{Year} variable can be questioned in two ways, thus providing justification for leaving it out: The data points of a certain district in 2012, 2013 and 2014 may not be fully independent of one another because it is likely that the conditions in that region have not substantially changed. The model seems unnecessarily complex, if we add additional smooth terms for each variable grouped on \texttt{Year} - it hardly brings any additional explantory power. Another possible violation of the independece assumption arises from spatial correlation - the fact that regions which are located closely to one another are mutually dependent of one another in terms of the number of TB cases as well as the socio-economic determinants of the spread of infectious diseases. Parametric coefficients of the \texttt{Year} factor in \texttt{temporal.model} barely differ from each other, at around $-8.4$ for \texttt{Year}=2012, and varying by $0.004$ and $-0.038$ respectively for 2013 and 2014, which respectively correspond to multipliers of 1.004 and 0.963 on the response scale.
\item The fitted vs. response plot shows a deviance from the 45-degree line for high absoulute values of the number of cases. One reason for this is that the model is designed to predict the ratio of TB cases per capita, not the absolute number. Another reason is that the model variance increases with the mean value.
\end{enumerate}

\section{Conclusions}
Firstly, based on the Correlogram we can conclude that no single socio-economic covariate has much linear correlation with TB incidence, but illiteracy, urbanisation, poverty, sanitation, unemployment and timeliness of notification are all weakly correlated with TB incidence, and given that there are more strong correlations between these socio-economic covariates, an increase in illiteracy, poverty, unemployment and Increases in illiteracy, poverty, unemployment and poor sanitation will simultaneously lead to decreases in urbanisation and timeliness of notification, ultimately leading to significant increases in TB incidence. Poverty is strongly correlated with several socio-economic covariates and is the primary factor that governments need to improve. Sanitation and urbanisation are also more strongly correlated, implying that good urban infrastructure and quality health resources have a greater impact on reducing TB incidence. According to our predicted TB incidence map, Brazil's central region has a lower incidence overall, so we recommend that the health sector invest significant health resources to improve the current situation throughout Brazil's north-west, followed by localised areas in the south and east, although these areas could also rely on assistance from neighbouring regions with fewer cases to improve their situation.
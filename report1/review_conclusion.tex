\section{Critical Review}

As demonstrated in Figure \ref{fig:spatial_model_2_check}, the QQ plots show deviations on the farthest quantiles, which demonstrate that the predictions do not cover the full range of the data. The independence assumption of the \texttt{Year} variable can be questioned wih notions of temporal and spatial correlation, thus providing justification for its exclusion. The data points of a certain region in 2012, 2013 and 2014 may not be fully independent of one another as it is likely that the conditions in that region have not substantially changed. The model seems unnecessarily complex if we add additional smooth terms for each covariate grouped on \texttt{Year} as it hardly brings any additional explanatory power. Parametric coefficients of the \texttt{Year} factor in \texttt{spatio.temporal.model.2} barely differ from each other at around $-8.4$ for \texttt{Year} = 2012, and varying by $0.004$ and $-0.038$ respectively for 2013 and 2014. These respectively correspond to multipliers of 1.004 and 0.963 on the response scale (See \texttt{summary(spatio.temporal.model.2)} in Appendix). Another possible violation of the independence assumption arises from spatial correlation. The regions which are located closely to one another are mutually dependent on one another in terms of the number of TB cases as well as the socio-economic determinants of the spread of infectious diseases.

\section{Conclusions}
Based on the correlogram, we can conclude that no single socio-economic covariate has much linear correlation with TB incidence, but illiteracy, urbanisation, poverty, sanitation, unemployment and timeliness of notification are all weakly correlated with TB incidence. There are stronger correlations between these socio-economic covariates. Increase in illiteracy, poverty, unemployment and poor sanitation will simultaneously correlate to decreases in urbanisation and timeliness of notification, all contributing to increases in TB incidence. Unsurprisingly, poverty is strongly correlated with several socio-economic covariates. Sanitation and urbanisation are also strongly correlated, implying that investments can be directed twoards improvement of urban infrastructure and health resources have a greater impact on reducing TB incidence. According to our predicted TB incidence map, Brazil's central region has a lower incidence overall. This presents an opportunity to learn from what has worked in this region and how improvements and resources can be further directed to Brazil's north-west state of Amazonas along with localised areas in the south and east, particularly in the southern and south-eastern states of Sao Paolo and Rio de Janeiro. It is worth noting that these areas of Sao Paolo and Rio de Janiero could benefit from assistance from neighbouring regions with fewer cases of TB (see Figure \ref{fig:brazil-division-states}).
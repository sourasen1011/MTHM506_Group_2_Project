\section{Introduction}

\subsection{Problem Statement}
This report contains an analysis of tuberculosis (TB) data originating from Brazil using Generalized Additive Models (GAMs). Brazil is divided into 557 administrative microregions, and the available data
contains counts of TB cases in each microregion for each of the years from 2012 to 2014.

\subsection{Exploratory Analysis of Data and the Problem}

The TB data from Brazil includes 1,671 entries or samples with 14 columns of numeric data types that specify the characteristics of each sample. The columns are: Indigenous, Illiteracy, Urbanisation, Density, Poverty, Poor Sanitation, Unemployment, Timeliness, Year, TB, Population, Region, lon, and lat. The columns `lon' and `lat' stand for longitude and latitude. The dataset has no missing values. The `region' is stored as a continuous variable despite being a factor variable. Nonetheless, changing it depends on the task at hand. The collection includes coordinates that describe the precise geospatial locations of the microregions listed. The next section gives a detailed exploration of  the data.

\subsection{Data Exploration}

Values quoted are in \% unless stated otherwise. An in-depth analysis of the datasets reveals that the mean and median values for the indigenous population are low, but the maximum value is 50, which suggests that there are individual areas where the indigenous population is concentrated. It is important to explore these areas to see if the poverty, sanitation and TB incidence differ significantly from the baseline. The mean and median illiteracy rates are only 14 and 11, respectively. However, the maximum value of 41 suggests that there are specific areas with significant parts of the population lacking access to education. This may suggest that the area has poor sanitation, but it needs further examination. There are still some areas that are less urbanised, where the TB occurrence may differ from the more urbanised areas. The median, and minimum values for urbanisation are 72 and 22, respectively, suggesting that most areas are highly urbanised.

Based on the population density data, most locations show that one person can have their individual room. But the highest value of 1.6 highlights some places with very high population densities, which sharply increase the rate of TB transmission. The distribution of the poverty data suggests that each district has different poverty levels, with only a limited number of districts where poverty is not a significant issue. Although the mean value for poor sanitation is around 13, the maximum score of 58 indicates that some districts have poor sanitation and substantial disease risk. Although average unemployment rates are low, a maximum value of 20 indicates that some isolated regions may experience severe economic hardship and potentially significant morbidity rates. With a minimum value of 0, notification timeliness data has a fairly wide range.

Timeliness, Unemployment, and Urbanization approximately follow a Normal Distribution with a few outliers, while the remainder consists of multimodal distributions. The above suggests that employing semi-parametric or non-parametric models to demonstrate the relationship between the target and predictors would be useful. The target variable in this study is a risk, defined as $\frac{TB}{Population}$, whereas the remaining variables are possible predictors. As demonstrated in Figure \ref{fig:spearman_correl}, most features in the dataset exhibit some correlaton. It is vital to note that some characteristics are anticipated to have positive correlations (tuberculosis versus population, density versus poverty) and vice-versa. Specifically, population density, poverty, health conditions, unemployment and notification timeliness are likely to be high due to the high population density, low economic share per capita, high poverty rate, and high unemployment rate. See Figure \ref{fig:spearman_correl} for a correlogram of the 8 socio-economic covariates.